\documentclass[../manuscript.tex]{subfiles}

\section{Обозначения, сокращения, основные определения}

\begin{definition}[\textit{Синергизм}]$ $\\
    Синергизм – тип взаимодействия между двумя или более химическими агентами, который характеризуется тем, что общий эффект лекарств превышает сумму индивидуальных эффектов каждого лекарства. 
\end{definition}

\begin{definition}[\textit{Хемоинформатика}]$ $\\
    Хемоинформатика это научная дисциплина на пересечении химии, информатики и математики. Основные задачи в области связанны с построением вычислительных методов для хранения и обработки химической информации и дизайна малых молекул с заданными свойствами.
\end{definition}

    
\begin{definition}[\textit{Дифференциальная экспрессия генов}]$ $\\
    Твое определение.
\end{definition}

\begin{definition} [\textit{(Q)SAR}]$ $\\
    Твое определение
\end{definition}