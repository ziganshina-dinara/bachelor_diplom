\documentclass[../manuscript.tex]{subfiles}

\section{Обозначения, сокращения, основные определения}

\begin{definition}[\textit{Синергизм}]$ $\\
    Синергизм – тип взаимодействия между двумя или более химическими агентами, который характеризуется тем, что общий эффект лекарств превышает сумму индивидуальных эффектов каждого лекарства. 
\end{definition}

\begin{definition}[\textit{Хемоинформатика}]$ $\\
    Хемоинформатика это научная дисциплина на пересечении химии, информатики и математики. Основные задачи в области связаны с построением вычислительных методов для хранения и обработки химической информации и дизайна малых молекул с заданными свойствами.
\end{definition}

    
\begin{definition}[\textit{Дифференциальная экспрессия генов}]$ $\\
    Явление дифференциальной экспрессии генов состоит в том, что экспрессия генов в одном клеточном состоянии отличается от их экспрессии в другом клеточном состоянии. Регуляция экспрессии генов может происходить на разных уровнях: репликации, транскрипции, трансляции, а также в процессе созревания иРНК и полипептидных цепей, образующихся в результате трансляции.
\end{definition}

\begin{definition}[\textit{Экспрессионная сигнатура}]$ $\\
    Набор статистически значимых дифференциально экспрессирующихся генов между 2 клеточными состояниями. Он состоит из 2 списков генов: с повышенной и пониженной экспрессией. 
\end{definition}

\begin{definition} [\textit{Quantitative Structure-Activity Relationship, (Q)SAR}]$ $\\
     Quantitative Structure-Activity Relationship - это одно из интенсивно развивающихся направлений использования математических методов в химии. Под QSAR подразумевается  поиск зависимостей между структурами химических соединений и их свойствами. Также часто аббревиатуру QSAR используют для обозначения моделей, в основе которых лежит концепция связи "структура-свойство" .
\end{definition}

\begin{definition} [\textit{Сигнальный путь}]$ $\\
     Сигнальный путь — это последовательность молекул, посредством которых информация от клеточного рецептора передается внутри клетки.
\end{definition}
