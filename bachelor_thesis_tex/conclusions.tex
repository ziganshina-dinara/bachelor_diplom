\documentclass[../manuscript.tex]{subfiles}
\section{Заключение. План дальнейших исследований}
В данной работе был реализован метод предсказания синергетических пар на основе данных RNA-seq. Было опробовано несколько способов вычисления уровня синергии пары малых молекул. Их валидация была проведена на задаче химического перепрограммирования. 


Первый способ основан на сравнении экспрессионных сигнатур с учетом топологии генных сетей. Одним из важных этапов его разработки был подбор оптимальных коэффициентов в метрике значимости гена. С этой целью для каждого рассматриваемого клеточного перехода были найдены коэффициенты с помощью байесовского оптимизатора. В качестве оптимальных значений использовались средние значения коэффициентов, определенных по всем переходам. Для проверки оптимальных коэффициентов было выполнено сравнение результатов валидации метода с использованием набора коэффициентов, полученных при байесовской оптимизации для каждого перехода, и оптимальных коэффициентов. При сравнении разницы средних значений уровня синергетических и несинергетических пар относительное снижение составляло не более 45\%. Для числа синергетических пар в топ 5\% понижение не превышало 5\%. На основе этого можно сказать, что оптимальные коэффициенты подходят под все рассмотренные клеточные переходы.


Второй способ основан на сравнении наборов обогащенных сигнальных путей. Для него было проверено 2 варианта вычисления уровня синергии, первый из которых основан на гипотезе комплементарности, второй на гипотезе сходства. Для этих методов была выполнена валидация, первый подход уступил по её результатам второму.


Если ориентироваться на разницу средних значений уровня синергии синергетических и несинергетических пар, то метод, основанный на сравнении экспрессионных сигнатур с учетом топологии генных сетей, показал наилучшие результаты. Если сравнивать долю синергетических пар в топ 5\%, то методы, основанные на сравнении экспрессионных сигнатур и сравнении обогащенных сигнальных путей, дали похожие результаты. 


В дальнейшем для улучшения метода планируется учитывать типы клеточных линий, на которых были индуцированы сигнатуры. Это необходимо из-за того, что клеточный ответ на возмущение малой молекулой сильно варьируется на разных клеточных линиях. Также планируется провести валидацию метода на других данных. 