\documentclass[../main.tex]{subfiles}

\section{Аннотация}

Синергизм – тип взаимодействия между двумя или более химическими агентами, который характеризуется тем, что общий эффект лекарств превышает сумму индивидуальных эффектов каждого лекарства. Задача вычислительного предсказания синергетических эффектов остается нерешенной и актуальной во многих областях биомедицины (например, для подбора комбинаций противоопухолевых лекарств или предсказания комбинаций для клеточного перепрограммирования). В работе представлен новый подход для предсказания синергетических комбинаций. Данный метод позволяет подбирать синергетические пары малых молекул для достижения интересующих изменений в фенотипе клетки. В качестве входных данных метод принимает экспрессионные сигнатуры, которые позволяют описать изменение в экспрессии генов между 2 различными клеточными состояниями. С помощью них можно охарактеризовать действие препарата на клеточную линию. В данной работе было опробовано несколько способов вычисления уровня синергии. 


Первый способ основан на сравнении экспрессионных сигнатур с учетом биологической значимости генов в контексте интересующих изменений. Значимость гена оценивается линейной комбинацией топологических метрик центральности, рассчитанных для него в генной сети. Была выполнена масштабная оптимизация метода по подбору коэффициентов этого выражения. 


Второй способ основан на сравнении наборов обогащенных сигнальных путей, которые также позволяют характеризовать действие препарата на клеточную линию. В данном способе было опробовано 2 варианта подсчета уровня синергии малых молекул. Первый основан на гипотезе, что синергетические малые молекулы, дополняя друг друга, затрагивают необходимые клеточные процессы. Во втором предполагалось, что синергетические пары соединений затрагивают одни и те же процессы.


Была выполнена валидация вышеупомянутых методов в задаче химического перепрограммирования клеток. 


